\chapter{Introduction} \label{introduction}

\section{Background}

\subsection{Graphs}

In this subsection, the main aspects of graph theory are briefly presented.

\subsubsection{Begginings and historical remarks}

Graph theory

\subsubsection{Introduction}

In the real world, many problems can be described by a diagram connecting a set of points with
lines, joining pairs of these points, or even creating loops on a single point. A simple example
of that would be a set of points representing people with lines connecting acquintances, or
points representing atoms and lines representing chemical bonds, creating a representation of
a molecule as a graph attribute. In the examples above, the only information contained is whether
two points are associated, with the manner being disregarded. The concept of a graph consists of
a mathematical abstraction of the above. \cite{book:2008}


\begin{definition} \label{u_simple_graph} Mathematically, in its simplest form, a
\textbf{undirected simple graph\footnote{}} is an ordered pair $G=(V, E)$ of:
\end{definition}

\begin{definition}\label{graph_def}
  A \textit{graph} G is an ordered pair \footnotemark{} $(V(G), E(G))$ consisting of a set
$V(G)$ of \textit{vertices} (also called \textit{nodes} or \textit{points}) and a set
$E(G)$, disjoint from $V(G)$ which consists of \textit{edges} (also called \textit{links}
or \textit{lines}) together with an incidence function $\psi_G$ that associates with each
edge of G an unordered pair of not necesserily distinct vertices of G.  If $e$ is an edge
and $u$ and $v$ are vertices such that $psi_G ={u, v}$ then $e$ is said to \textit{join}
$u$ and $v$, and the vertices $u$ and $v$ are called the \textit{ends} of $e$. We denote
the numbers of vertices and edges $G$ by $u(G)$ and $e(G)$ which two parameters are called
the \textit{order} and \textit{size} of G, respectively \cite{book:2008}.

  In short, we can define a \textbf{graph} as an ordered triple $G=(V, E, \phi_G)$
consisting of:
  \begin{itemize}
  \item $V$, a set of \textit{vertices}
  \item $E$, a set of \textit{edges}
  \item $\phi_G: E \rightarrow \{\{x, y\} | x, y \in V \; and \; x \neq y\}$ an
\textit{incidence function} mapping every edge to an unordered pair of vertices - an edge
associated with two distinct vertices.
  \end{itemize} This type of object is called an \textit{undirected multigraph}, to avoid
confusion. Note, that the above definition of the \textit{incidence function} does not
allow for \textit{loops} (mappings of an edge on the same vertex).

A \textit{loop} is a an edge that allows a connection of a vertex to itself and a graph can
be defined to either allow or disallow the presence of loops. Some authors allow for loops
to exist on \textit{multigraphs} \cite{article:bollobas}, while other consider these kind
of graphs to exist in a different category, called \textit{pseudographs} \cite{book:Gary}.
Allowing loops requires modifying the incidence function so they can be supported. The new
incidence function can be written as:
\begin{equation*} \phi_G : E \rightarrow \{\{x, y \} | x,y \in V\} \end{equation*}
Graphs which su
  



\footnotetext{An ordered pair $(a, b)$ is a pair of objects in which the order of
appearance or insertion is significant; the ordered pair $(a, b)$ is different than $(b,
a)$ unless $a=b$. An unordered pair is a set of the form ${a, b}$ is a set having two
elements with no relation between them and ${a, b} = {b, a}$.  }

\end{definition}



